% !TEX root = digitale-bedingungen.tex

\documentclass[headsepline,12pt,a4paper]{scrreprt}
\renewcommand{\baselinestretch}{1.6}\normalsize %Zeilenabstand 1,5 Zeilig (baseline um 1.25 multipliziert).
\renewcommand{\footnotesize}{\fontsize{9}{10}\selectfont} %Anpassung der Schriftgröße in den Fußnoten
\usepackage[left=3cm, right=2.5cm, bottom=3cm, top=2.5cm]{geometry}
	\setlength{\parindent}{0cm}
\setcounter{tocdepth}{5}
\setcounter{secnumdepth}{5}
\usepackage[ngerman]{babel}
\usepackage[T1]{fontenc}
\usepackage[utf8]{inputenc} 
\usepackage[fencedCode,inlineFootnotes,citations,definitionLists,hashEnumerators,smartEllipses,hybrid]{markdown}
%% You can re-define how links are rendered: uncomment the following to get hyperlinked text instead of footnotes
% \markdownSetup{rendererPrototypes={
%   link = {\href{#2}{#1}},
%   image = {\begin{figure}[hbt!]
%     \centering
%     \includegraphics{#3}%
%     \ifx\empty#4\empty\else
%     \caption{#4}%
%     \fi
%     \label{fig:#1}%
%     \end{figure}}
% }}

\usepackage{tabularx}
\usepackage{docmute} % Ermöglicht das Kompilieren aus einzelnen Kapiteln heraus, nicht nur aus der Hauptdatei des Buches.
\usepackage{csquotes} %Paket zum Einfügen von mehrzeiliger Zitate mit dem Befehl displayquote
%   \renewenvironment{displayquote} %Rechtsbündige mehrzeilige Zitate
%    {\small\list{}{\rightmargin=0cm \leftmargin=1.5cm}
%      \item\relax}
%     {\endlist}
\usepackage{framed,xcolor}
\colorlet{shadecolor}{red!25}
\usepackage{graphicx}
\usepackage{chngcntr} %Fortlaufende Nummerierung der Abbildungen nicht nach Kapitel 
  \counterwithout{figure}{chapter} %Gehört zu den fortlaufenden Nummerierungsoptionen 
\usepackage{color}
\usepackage{setspace}
\definecolor{mygreen}{rgb}{0,0.6,0}
\definecolor{mygray}{rgb}{0.5,0.5,0.5}
\definecolor{mymauve}{rgb}{0.58,0,0.82}
\definecolor{orange}{rgb}{0.255,0.153,0}
\usepackage[final]{pdfpages}
\usepackage{fancyvrb}
\usepackage{listings}
\lstset{numbers=left, numberstyle=\tiny, numbersep=5pt}
\lstset{language=PHP}
\lstset{ %
  backgroundcolor=\color{white},   % choose the background color; you must add \usepackage{color}
  basicstyle=\footnotesize,        % the size of the fonts that are used for the code
  breakatwhitespace=false,         % sets if automatic breaks should only happen at whitespace
  breaklines=true,                 % sets automatic line breaking
  captionpos=b,                    % sets the caption-position to bottom
  commentstyle=\color{mygreen},    % comment style
  deletekeywords={default},            % if you want to delete keywords from the given language
  escapeinside={\%*}{*)},          % if you want to add LaTeX within your code
  extendedchars=true,              % lets you use non-ASCII characters; for 8-bits encodings only, does not work with UTF-8
  frame=single,                    % adds a frame around the code
  keepspaces=true,                 % keeps spaces in text, useful for keeping indentation of code (possibly needs columns=flexible)
  keywordstyle=\color{blue},       % keyword style
  language=Octave,                 % the language of the code
  morekeywords={*,...},            % if you want to add more keywords to the set
  numbers=left,                    % where to put the line-numbers; possible values are (none, left, right)
  numbersep=5pt,                   % how far the line-numbers are from the code
  numberstyle=\tiny\color{white}, % the style that is used for the line-numbers
  rulecolor=\color{white},         % if not set, the frame-color may be changed on line-breaks within not-black text (e.g. comments (green here))
  showspaces=false,                % show spaces everywhere adding particular underscores; it overrides 'showstringspaces'
  showstringspaces=false,          % underline spaces within strings only
  showtabs=false,                  % show tabs within strings adding particular underscores
  stepnumber=2,                    % the step between two line-numbers. If it's 1, each line will be numbered
  stringstyle=\color{mymauve},     % string literal style
  tabsize=2,                       % sets default tabsize to 2 spaces
  title=\lstname                   % show the filename of files included with \lstinputlisting; also try caption instead of title
}
\lstdefinestyle{customc}{
  aboveskip=1.5\baselineskip,  
  belowcaptionskip=0\baselineskip,
  breaklines=true,
  frame=L,
  xleftmargin=\parindent,
  language=C,
  showstringspaces=false,
  basicstyle=\footnotesize\ttfamily,
  keywordstyle=\bfseries\color{orange},
  commentstyle=\itshape\color{purple!40!black},
  identifierstyle=\color{blue},
  stringstyle=\color{orange},
}
\lstset{escapechar=@,style=customc}
\lstset{literate=%
{Ö}{{\"O}}1 
{Ä}{{\"A}}1 
{Ü}{{\"U}}1 
{ß}{{\ss}}2 
{ü}{{\"u}}1 
{ä}{{\"a}}1 
{ö}{{\"o}}1
}
\usepackage[justification=raggedright,singlelinecheck=false]{caption} %Setzt die Bildunterschriften linksbündig

\usepackage{lastpage} % for the number of the last page in the document
\usepackage{fancyhdr} % Eigene Kopfzeile erstellen
\pagestyle{fancy}
\lhead{\nouppercase{\leftmark}} %Kopfzeile mit Kapitelnummer und Kapitelbezeichnung
\chead{}
\rhead{}
\usepackage{pslatex}
\usepackage{booktabs} %Erweiterte Optionen für Tabellen (toprule, midrule, bottomrule)
\usepackage{multirow} %Erweiterte Optionen für Tabellen (multirow, multicolumn)
\usepackage{lineno}
%Ermöglicht fortlaufende Fußnoten über Kapitel hinweg
\usepackage{remreset}
\makeatletter
\@removefromreset{footnote}{chapter}
\makeatother

\usepackage{minitoc}


\usepackage{pdfpages}  %Befehl zum Einbinden \includepdf[pages={5,8,10-14}]{internal_rate_of_return.pdf}

\usepackage{ragged2e} %Anpassung des Rausatzes im Literaturverzeichnis

\usepackage{jurabib}

\jurabibsetup{
	authorformat=dynamic,
	authorformat=smallcaps, % Autoren werden in Kapit\"alchen gesetzt
	%authorformat=and, % Autoren werden durch 'und' getrennt
	authorformat=year, % Das Jahr wird in den Zitaten mit angegeben
	% bibformat=ibidem, %Ebd. wird auch im Literaturverz. angegeben
	bibformat=raggedright,
	%authorformat=italic, % Der Autor wird in den Zitaten kursiv gesetzt
	commabeforerest, % Komma vor der Seitenzahl und erg\"anzenden Eintr\"agen
	%citefull=first, %Beim ersten Zitat wird der komplette Bib-Eintrag angezeigt
	dotafter=bibentry, % setzt einen Punkt nach dem Bib-Eintrag
	%titleformat={colonsep}, % Doppelpunkt zwischen Autor und Titel
	ibidem=strict, %Erstellt ebd. und A.o.O bei Wiederholungen
	% nopublisher=both,
	titleformat=italic, % setzt den Titel kursiv
	%titleformat=all, % der gesamte Titel wird im Zitat angezeigt
	% titleformat=ibidem,
	%chicago,
	%oxford,
	pages=format, % vor den Seitenzahlen erscheint ein 'S.'
	%bibformat=ibidemalt % Eine alternative Darstellung des Literaturverzeichnisses,
	%bibformat=compress, % Der Abstand zwischen den einzelnen Beitr\"agen wird verringert
	%super, %konvertiert alle \cite-Befehle in \footcite
	see=true,
	round,
	% edby=true,
	% natoptargorder,
	%authorformat=allreversed, %\"Anderungen der Zitate gelten auch f\"ur das Lit-Verzeichnis 
	%howcited=normal %Zitiert nach erscheint hinter dem Titel
}
%\usepackage[bottom, hang]{footmisc}%Erweiterte Einstellungen für die Fußnoten
%\setlength{\footnotemargin}{5pt} %Einzug des Fußnotentextes
%\setlength{\footnotesep}{<neuerAbstand>}%Abstand zwischen den Fußnoten
%\setlength{\skip\footins}{<neuerAbstand>}% Abstand von Haupttext und Fußnoten

\deffootnote[1em]{1em}{0em}{\textsuperscript{\thefootnotemark}}%KOMA Fußnoten Einstellungen

% \DeclareRobustCommand{\jbissn}[1]{}
% \DeclareRobustCommand{\jbisbn}[1]{}
% für Ebd.:
\AddTo\bibsgerman{%
\renewcommand*{\ibidemname}{Ebd.}
\renewcommand*{\ibidemmidname}{Ebd.}
\renewcommand*{\urldatecomment}{Stand: } %Date Access von Mendeley Anzeige
\def\incollinname{In:}
\def\editorname{(\textup{Hrsg.})}%
\def\editorsname{(\textup{Hrsg.})}%
\def\etalname{\unskip\nobreakspace{}\textup{et\,al .}}%
\def\etalnamenodot{\unskip\nobreakspace{}\textup{e t\,al}}%
% \def\edbyname{herausgegeben von}%
% \def\Edbyname{Herausgegeben von}%
}

%Formatierung URL
%\biburlfont{tt} % typewriter 
\biburlfont{rm} % roman 
%\biburlfont{sf} % serifenlos 
%\biburlfont{same} % wie im Text

% Formatierung der Zitate
\renewcommand{\jbcitationyearformat}[1]{#1}%
\renewcommand*{\jbbtasep}{; } 
\renewcommand*{\jbbfsasep}{, } 
\renewcommand*{\jbbstasep}{; }
% \renewcommand*{\jbbtesep}{ \& } 
% \renewcommand*{\jbbfsesep}{, } 
% \renewcommand*{\jbbstesep}{ \& }

% Formatierung der Bibliography Angaben

\renewcommand*{\bibbtasep}{; } % bta = between two authors sep
\renewcommand*{\bibbfsasep}{; } % bfsa = between first and second author sep
\renewcommand*{\bibbstasep}{; }% bsta = between second and third author sep

\renewcommand*{\bibbtesep}{; } % bte = between two editors sep
\renewcommand*{\bibbfsesep}{; } % bfse = between first and second editor sep
\renewcommand*{\bibbstesep}{; }% bste = between second and third editor sep

\renewcommand*{\biburlprefix}{[URL: } %Eintrag vor der URL
\renewcommand*{\biburlsuffix}{]} %Eingtrag nach der URL
\biburlfont{same}

\usepackage[pagebackref]{hyperref} %Verlinkungen innerhalb und außerhalb des Dokumentes und zusätzlich Backlinks zu den Fußnoten

\hypersetup{
    bookmarks=true,         % show bookmarks bar?
    unicode=false,          % non-Latin characters in Acrobat’s bookmarks
    pdftoolbar=false,        % show Acrobat’s toolbar?
    pdfmenubar=true,        % show Acrobat’s menu?
    pdffitwindow=true,     % window fit to page when opened
    pdfstartview={FitH},    % fits the width of the page to the window
    pdftitle={Masterarbeit: Lernen? Potenziale aktueller Medienkultur am Beispiel einer digitalen Mitschriftassistenz},    % title
    pdfauthor={Manfred Steger},     % author
    pdfkeywords={Universität Hamburg} {Masterarbeit} {Lerntheorie} {Aktuelle Medien} {Medientheorie} {Schule} {Inklusion} {Inklusion Medien} {Mitschriftassistenz}, % list of keywords
    pdfnewwindow=true,      % links in new window
    colorlinks=true,       % false: boxed links; true: colored links
    linkcolor=black,          % color of internal links (change box color with linkbordercolor)
    citecolor=black,        % color of links to bibliography
    filecolor=black,      % color of file links
    urlcolor=black           % color of external links
}
\usepackage{subfigure}
\usepackage{titlesec} % Textüberschriften anpassen
% \titleformat{Überschriftenklasse}[Absatzformatierung]{Textformatierung} {Nummerierung}{Abstand zwischen Nummerierung und Überschriftentext}{Code vor der Überschrift}[Code nach der Überschrift]

%\titleformat{\chapter}[hang]{\Large\bfseries}{\thechapter\quad}{20pt}{}
%\titleformat{\section}[hang]{\large\bfseries}{\thesection\quad}{20pt}{}
%\titleformat{\subsection}[hang]{\large\bfseries}{\thesubsection\quad}{20pt}{}
%\titleformat{\subsubsection}[hang]{\large\bfseries}{\thesubsubsection\quad}{20pt}{}
%\titleformat{\paragraph}[hang]{\large\bfseries}{\theparagraph\quad}{20pt}{}

% \titlespacing{Überschriftenklasse}{Linker Einzug}{Platz oberhalb}{Platz unterhalb}[rechter Einzug]

%\titlespacing{\chapter}{0pt}{10pt}{10pt}[10pt]
\titlespacing{\section}{0pt}{25pt}{10pt}
%\titlespacing{\subsection}{0pt}{20pt}{10pt}
%\titlespacing{\subsubsection}{0pt}{20pt}{10pt}
%\titlespacing{\paragraph}{0pt}{20pt}{10pt}

\usepackage{pdfcomment} % Create Tooltips/Alt Text for Images


%%%%%%%%%Deckblatt%%%%%%%%%%%


\clubpenalty = 10000 
\widowpenalty = 10000 

%%%%%% Dieser Absatz verhindert den Seitenumbruch bei Kapiteln
%\makeatletter 
%\renewcommand\chapter{\thispagestyle{plain}%
%\global\@topnum\z@
%\@afterindentfalse
%\secdef\@chapter\@schapter}
%\makeatother 
%%%%%%