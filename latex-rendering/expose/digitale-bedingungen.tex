\input{../praambel/praambel} % bindet die Präambel ein

\begin{document}

\input{../deckblatt/deckblatt} % bindet das Deckblatt ein

\input{../inhaltsverzeichnis/inhaltsverzeichnis}

\chapter{Problemstellung}

Hallo mein name  % Kapitel: Problemstellung

\chapter{Forschungsstand}

Ein wenig Text
 % Kapitel: Aktueller Forschungsstand

\chapter{Forschungslücke}

neuer 
 % Kapitel: Identifizierung der Forschungslücke

\input{../../forschungsziel} % Kapitel: Forschungsziel

\input{../../fragestellung} % Kapitel: Zentrale Fragestellungen und Hypothesen

\input{../../relevanz} % Kapitel: Relevanz der Arbeit

\input{../../forschungsdesign} % Kapitel: Forschungsdesign

\input{../../analysekorpus} % Kapitel: Untersuchungsmaterial / Analysekorpus

\input{../../aufbau} % Kapitel: Aufbau der Arbeit

\input{../../zeitplan} % Kapitel: Zeitplan und einzelne Arbeitsschritte

\input{../../auswahlbibliografie} % Kapitel: Auswahlbibliografie

\input{../eidesstattlich/eidesstattlich} % Eidesstattliche Erklärung

\end{document}
